\documentclass[a4paper]{article}

\title{Teaching a computer to learn, find and read licence plates}
\date{November 17th, 2011}

% Paragraph indentation
\setlength{\parindent}{0pt}
\setlength{\parskip}{1ex plus 0.5ex minus 0.2ex}

\begin{document}
\maketitle

\section*{Project members}
Gijs van der Voort\\Richard Torenvliet\\Jayke Meijer\\Tadde\"us Kroes\\Fabi\'en Tesselaar

\tableofcontents
\setcounter{secnumdepth}{1}

\section{Introduction}

Licence plates are used all over the world. The plates are, usually, attached to the front and rear
of a motorised vehicle and used for identifying this vehicle. Every
country can have more or less its own version of a licence plate, but all these systems do not 
differ greatly. We will be focusing on the Dutch system for licence plates.

\section{Problem Description}

License plates are used for identification and thus made to recognize from great
distances and still be seen in many weather conditions. Our learning set of photos contains
'' ik weet niet precies wat voor camera ''. The angle in which these pictures are taken or the angle 
of the approaching vehicles are always different and some licence plates are a bit dirty,
but for a human they are still pretty easy to identify. A computer or perhaps a small
chipset will need to be thoroughly practiced. In short our program must be able to
do the following:

\begin{enumerate}
\item Find the location of the license plate.
\item Use perspective transformations to obtain an upfront view.
\item Reduce noise where possible.
\item Find the locations of each letter and extract it.
\item Apply a Local Binary Pattern algorithm on each letter.
\item Match the found patterns with results from the learning set and return the best match for each letter.
\end{enumerate}

\section{Solution}

Now that we know the problem we can start with stating our solution. This will
come in a few steps as well.

\subsection{Localizing the plate}

The photos are of very high contrast. Most of the time only the lights of a vehicle
are visible in addition to the license plate. We can first crop the image until
it finds brighter pixel values in a row or column. Then we can apply ''?? weet niet hoor'' local histogram
matching to find out whether we have a light or license plate.

\subsection{Transformations}

Once the locations of the four corner points of the license plate have been
found, a simple perspective transformation will be sufficient to transform and
resize the plate to a normalized format.

\subsection{Reducing noise}

Weet niet precies hoe, maar van die kleine rondjes / vlekjes / stipjes moeten
we wel een beetje weghalen want die maken het wel een beetje lelijk

Taddeus: Ik brainstorm hier een beetje...:

Small amounts of noise will probably be suppressed by usage of a Gaussian
filter. A real problem occurs in very dirty licence plates, where branches and
dirt over a letter could radically change the local binary pattern. A question
we can ask ourselves here, is whether we want to concentrate ourselves on
these exceptional cases. By law, license plates have to be readable.
Therefore, we will first direct our attention at getting a higher score in the
'regular' test set before addressing these cases.

\subsection{Extracting a letter}

De karakteristiek bepalen van het dash/streepje (-) dan heb je in elk geval al
drie groepen met maar 1 of 2 letters (ws 2). Hier kun je volgens mij dan wel 
makkelijk zoeken op een overgang van letter naar andere letter omdat er stuk
white-space tussenzit

\subsection{Local binary patterns}

Hier moet een vrij groot verhaal omdat dit ons belangrijkste algoritme moet zijn

+ not sure if it will work out :o

\subsection{Matching the database}

Als we al die histogrammen opslaan, hoe gaan we dat slim met elkaar vergelijken
(of naja sneller dan brute force)


\section{Conclusion}

This will be fun.

\end{document}
